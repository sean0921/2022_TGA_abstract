\documentclass[a4paper]{article}
%\usepackage{simplemargins}

%\usepackage[square]{natbib}
\usepackage{amsmath}
\usepackage{amsfonts}
\usepackage{amssymb}
\usepackage{graphicx}

%% --Font style setting--
\usepackage{indentfirst}
\usepackage{mathspec}
\setmainfont{Liberation Serif}
\setmathsfont{Liberation Serif}

%CJK
\usepackage{xeCJK}
\setCJKmainfont{全字庫正楷體}

\begin{document}
\pagenumbering{gobble}

\Large
 \begin{center}
藉由區域地震訊號探討臺南外海群震行為與地下構造\\ 

\hspace{10pt}

% Author names and affiliations
\large
何立新$^1$, 饒瑞鈞$^1$ \\

\hspace{10pt}

\small  
%$^1$) 國立成功大學地球科學系\\
(1) 國立成功大學地球科學系\\
%sean.li.shin.ho@gmail.com\\

\hspace{10pt}


\large
\textbf{摘  要}
\end{center}

\normalsize
位於臺南濱海且延伸至外海的變形前緣,相對於鄰近西部麓山帶構造活動較為寧靜,構造以切穿地殼尺度的張裂斷層為主,帶走滑分量。在美濃地震前後,該區地震活動開始變得較爲頻繁,在2017年2月10日發生 $M_{L}$ 5.7的主餘震序列後達到高峰。2020年3月26日,鄰近區域也發生了規模達 $M_{L}$  3.2的群震序列。然而附近的氣象局寬頻站分佈較為缺乏,以致欲確認區域近海地震定位結果,進而探討鄰近構造活動等有所限制。

我們利用2020年2月底至5月底於臺南市區架設之35部CUBE短週期地震儀密集陣列收集之資料,定出3月26日之群震事件,並使用同期間中正大學於嘉南地區架設21個寬頻地震站、氣象局架設之強震站資料提升測站包覆性,增加定位品質可信度。

本研究挑出41個地震事件,最大規模達 $M_{L}$ 3.42,最小規模 $M_{L}$ 1.78。搭配Huang, 2014的三維速度模型,進行非線性定位估計其誤差後,可以發現藉由氣象局到時推算出的ERH、ERZ分別介於0.8 - 2.8公里、0.5 - 2.3公里之間,而我們的結果則介於0.5~1.4公里、0.2 - 1.1公里之間,顯示我們的資料能對改善該群地震事件之定位誤差。

進行GrowClust相對定位後,ERH、ERZ皆落在0.9公里內,並發現深度主要集中在14 - 15公里,可看到走向呈北北東約10度,傾角約向北傾45度的線性排列。利用上下動分佈分析群震中初動訊號較明顯的4組震源機制解,發現機制解多偏向正斷層帶走滑分量,可與近年內中研院AutoBATS記錄鄰近地區之機制解結果相吻合。另發現群震中規模較大事件的地震波形資料在P波到時後立刻收到特殊波相,波相速度相較P波略為緩慢但明顯快於S波。分析其波相的偏振方向並與P波相比,其入射與垂直方向夾角分佈較小,推測其為莫荷面反射訊號。利用波線追蹤方法比對其到時,推算其深度為18公里。

\end{document}
